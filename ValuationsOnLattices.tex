
\documentclass[12pt]{amsart}
\usepackage{geometry} % see geometry.pdf on how to lay out the page. There's lots.
\geometry{a4paper} % or letter or a5paper or ... etc
% \geometry{landscape} % rotated page geometry

% See the ``Article customise'' template for come common customisations

\title{Notes for Statistical Mechanics of Partitions}
\author{}
\date{} % delete this line to display the current date

%%% BEGIN DOCUMENT
\begin{document}

\maketitle
%\tableofcontents

%%%%%
%%%%%%
%%
%%  V A L U A T I O N S
%%
\section{Valuations on Lattices}
Oct 17, 2018

\subsection{Example: Buff	on's needle problem}
\subsection{GC. Rota's seminar\cite{GCRota97}:  Valuations on Lattices}
$\,$\linebreak 
%Indicator function
{\bf Definition 1}: If $A\subset \Sigma$, for a set of ($N_\Sigma$) elements, $\Sigma$, the {\sl indicator function},
or simply indicator, of $A$, denoted as $I_A$, is the function on $\Sigma$ given by
\[
\begin{array}{c}
I_A(c) \,=\,1;\quad c\,\in\,A 
\\
I_A(c) \,=\,0;\quad c\,\notin\,A 
\end{array}
\]

{\bf Note to self:} For partitions, the indicator function would be defined for the set of {\sl blocks} or
{\sl clusters}. Hence, if $S$ is the underlying set of $N$ elements whose partitions we are studying, 
\[
N_\Sigma\,=\,\sum_{j=1}^N\,\left(\begin{array}{c}N\\j\end{array}\right)\,=\,2^N\,-\,1
\]
and $\Sigma\,=\, \mathcal{P} (S)\,-\,\emptyset$ the set of all non-empty subsets of $S$. Any partition $P$ can be viewed as $P=\{P_1,\dots,P_\mathcal{K}\}\,\subset\,\Sigma(S)$, a finite collection of non-empty subsets, $P_i \subset S$ such that they cover $S$, i.e., $S=\cup_i\,P_i$. 
Hence, $\forall\, P\in\Pi(S)\,\Rightarrow \,P\subset \Sigma(S)$.

Indicator functions satisfy:
\[
\begin{array}{c}
I_{A\wedge B}\,=\,I_AI_B\\
I_{A\vee B}\,=\,I_A\,+\,I_B\,-I_AI_B\,=\,1-(1-I_A)(1-I_B)
\end{array}
\]

(\textbf{TODO}: Check that, indeed, we can substitute $\cap,\,\cup$ for $\wedge,\,\vee$) 

%Simple Function
\textbf{Definition 2}: An \textit{L-simple} function, or \textit{simple} function, is a finite  linear combination 
\[
f\,=\,\sum_{i=1}^k\,\alpha_i\,I_{{}_iP}
\]
where $\alpha_i\in\mathbb{R}$ and  ${}_iP\in\Pi(S)$ are partitions of a given set $S$.
 
%Valuation
{\bf Definition 3}: A valuation $\mu$ on the lattice, $\Pi$, of all partitions of a set S is a function
\[ 
\begin{array}{c}
\mu \, : \, \Pi \, \longrightarrow \, \mathbb{R}  
\\
\forall \, P\,,Q\,\in\,\Pi \, \rightarrow \, \\
\mu(P\vee Q) \,=\, \mu(P)\,+\,\mu(Q)\,-\,\mu(P\wedge Q)
\\
\mu(\bar{0}) \,=\,0 
\end{array}
\]
For distributive lattices $P\wedge(Q\vee R) = (P\wedge Q)\vee (P\wedge R)$ (and its dual relation).

Iterating
\[
\begin{array}{l}
\mu(P\vee Q\vee R)\,=\,\mu(P)\,+\,\mu(Q\vee R)\,-\,\mu( (P\wedge Q)\vee (P\wedge R))\,=\,
\\
\mu(P)\,+\,\mu(Q)\,+\,\mu(R)\,-\,\mu(P\wedge Q)\,-\,\mu(P\wedge R)\,-\,\mu(Q\wedge R)\,+\,
\mu(P\wedge Q\wedge R)\\
\\
\mu(P\vee Q\vee R\vee T\vee \dots) \,=\,

\mu(P)\,+\,\mu(Q)\,+\,\mu(R)\,+\,\mu(T)\,\dots + \\
-\,\mu(P\wedge Q)\,-\,\mu(P\wedge R)\,-\,\dots \,+ \\
+\,\mu(P\wedge Q\wedge R)\,+\,\mu(P\wedge Q\wedge T)\,+\,\dots\,-\\
-\,\mu(P\wedge Q\wedge R\wedge T)

\end{array}
\]

%Generating set of L
\textbf{Definition 4}: A \textit{generating set} of $\Pi(S)$ is a subset $G\subset \Pi(S)$, such that $\forall P\in\Pi \Rightarrow P=\vee_i\, B_i\quad; \,B_i\in G$.

Using the inclusion-exclusion formula for indicators, it can be shown that any simple function can be written as 
\[
f\,=\,\sum_{i=1}^r\,\beta_i\,I_{B_i}
\]

A valuation $\nu$ on $G$ can be extended to one $\mu$ on $\Pi(S)$ by using the exclusion-inclusion formula. $\forall\,P\in\Pi(S)\;;\;P=B_1\vee\dots\vee B_n$
\[
\mu(P)\,=\,\sum_i\,\nu(B_i)\,-\,\sum_{i<j}\,\mu(B_i\wedge B_j)\,+\,\dots
\]

%Integral wrt \mu on G
\textbf{Definition 5}: Given a valuation $\mu$ on $G$, and a simple function 
$f=\alpha_1\,I_{B_1}\,+\,\dots\,+\,\alpha_kI_{B_k}$, with $B_i\in G$, we define the
\textit{integral of $f$ wrt $\mu$} as 
\[
\int\,f\,d\mu\,=\,\sum_{i=1}^k\,\alpha_i\,\mu(B_i)
\] 

\textbf{Groemer's Integral Theorem}: Let $G$ be a generating set for a lattice $L$ and $\mu$ a valuation on $G$. The following statements are equivalent
\begin{enumerate}
\item $\mu$ extends uniquely to a valuation on $L$.
\item $\mu$ satisfies the inclusion-exclusion identities
\[
\mu(B_1\vee\dots\vee B_n)\,=\,\sum_i\,\mu(B_i)\,-\,\sum_{i<j}\,\mu(B_i\wedge B_j)\,+\,\dots
\]
\item $\mu$ defines an integral on the vector space of linear combinations of indicator functions of sets in $L$

Proof: (1)$\Rightarrow$ (2) and (3)$\Rightarrow$(1), trivial; (2)$\Rightarrow$(3), not trivial.

\end{enumerate}

\subsection{Valuations on Simplicial Complexes}
\subsection{Euler Characteristic}

\bibliographystyle{plain}
\begin{thebibliography}{99}
\bibitem{GCRota97} {\sl Introduction to Geometric Probability}, Daniel A. Klain and Gian-Carlo Rota, CUP, 1997.
\end{thebibliography}

\end{document}









